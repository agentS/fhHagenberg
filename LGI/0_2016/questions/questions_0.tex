\documentclass[a4paper]{article}
\usepackage{ucs}
\usepackage[utf8x]{inputenc}
\usepackage[T1]{fontenc}
\usepackage{german}
\usepackage{a4,ngerman}
\usepackage[ngerman]{babel}
\usepackage{graphicx}
\usepackage[]{cite}
\usepackage{fancyhdr}
\pagestyle{fancy}
\selectlanguage{german}
\usepackage{array}
\usepackage{mathtools}
\cfoot{\textcopyright Lukas Schörghuber, S1610307103}

\begin{document}
	\section{Kontrollfragen}

	\begin{enumerate}
		\item
		\begin{enumerate}
			\item Was sind Aussagen im Sinn der Aussagenlogik?
			\newline
			Eine Aussage ist ein Satz, der entweder wahr oder falsch ist.
			
			\item Worin besteht der Unterschied zwischen atomaren und zusammengesetzten Aussagen?
			\newline
			Atomare Aussagen stellen einen Wahrheitswert dar und lassen sich nicht mehr zerlegen.
			Zusammengesetzte Aussagen bestehen aus atomaren Aussagen, die mittels Junktoren miteinander verbunden sind.
		\end{enumerate}
		
		\item
		\begin{enumerate}
			\item 3 wohlgeformte Aussagen:
			\begin{equation*}
				(\lnot A) \lor (B \Rightarrow C)
			\end{equation*}
			\begin{equation*}
				((x \lor z) \land y) \Rightarrow (\lnot x \lor z)
			\end{equation*}
			\begin{equation*}
				(\lnot(\alpha \otimes \gamma)) \lor (\beta \land \lnot\gamma)
			\end{equation*}
			
			\item 3 nichtwohlgeformte Aussagen:
			\begin{equation*}
				\lnot A B \Rightarrow C
			\end{equation*}
			\begin{equation*}
				(((x \lor z) \land y)(\lnot x \lor z))
			\end{equation*}
			\begin{equation*}
				((((\lnot(\alpha \otimes \gamma)) \lor (\beta \lnot\gamma))))
			\end{equation*}
			
		\end{enumerate}
		
		\item
		Die Syntax legt fest, welche Regeln eine Aussage erfüllen muss, um als wohldefiniert zu gelten. Die Semantik legt fest, welche Bedeutung eine Aussage hat.
		
		\item
		Um als wohlgeformt zu gelten, muss eine Aussage die folgenden Regeln erfüllen:
		\begin{enumerate}
			\item Die Zeichen A, B, C, ... sind atomare Aussagen
			
			\item Wenn $\alpha$ und $\beta$ Aussagen sind, dann sind auch folgende Zeichenketten Aussagen:
			\begin{equation*}
				(\lnot \alpha)
			\end{equation*}
			\begin{equation*}
				(\alpha \land \beta)
			\end{equation*}
			\begin{equation*}
				(\alpha \lor \beta)
			\end{equation*}
			\begin{equation*}
				(\alpha \Rightarrow \beta)
			\end{equation*}
			
			\item Es gibt ebensoviele öffnende wie schließende Klammern
			\item links und rechts der Junktoren $\land$, $\lor$, $\Rightarrow$ stehte eine Aussage
			\item Eine Aussage endet nie mit einem Junktor
		\end{enumerate}
		
		\item
		Die Semantik zusammengesetzter Aussagen kann ermittelt werden, indem mittels einer Wahrheitstabelle die Semantik der Teilaussagen (können sowohl zusammengesetzte als auch atomare Aussagen sein) bestimmt wird. Die Teilaussagen sind dabei solange rekursiv zu zerlegen bis zwei atomare Wahrheitswerte konjungiert, disjunktiert oder impliziert werden oder ein atomarer Wahrheitswert negiert wird. Danach werden die Junktoren zwischen den Teilaussagen ausgewertet.
		
		\item
		Die Disjunktion ist auch wahr, wenn beide Aussagen wahr sind, während im Sprachgebrauch meistens das exklusive Oder verwendet wird, welches ausschließt, dass beide Aussagen gleichzeitig wahr sind.
		
		\item
		Der Nachweis kann mittels Einsetzen in Wertetabellen erbracht werden. Richtiges Umformen bedeutet, dass die zwei verglichenen Aussagen äquivalent sind.
		
		\item
		Die Verwendung von $=$ impliziert, dass die einzelnen Teilaussagen in der Wahrheitstabelle ebenfalls gleich sind und nicht nur das endgültige Resultat. $\equiv$ impliziert, dass nur das Endergebnis gleich sein muss.
		\newline
		Weiters sagt die Verwendung von $=$ etwas über das Ergebnis der Aussage aus, während $\equiv$ eine Meta-Aussage über die linke und die rechte Aussage darstellt.
		
		\item
		L ... Peter ist gut gelaunt
		\newline
		M ... Mittagessen hat geschmeckt
		
		\begin{tabular}{|l|l|c|c|}
			\firsthline
			L & M & M $\Rightarrow$ L & L $\Rightarrow$ M \\
			\hline
			f & f & \textbf{w} & \textbf{w} \\
			f & w & \textbf{f} & \textbf{w} \\
			w & f & \textbf{w} & \textbf{f} \\
			w & w & \textbf{w} & \textbf{w} \\
			\hline
		\end{tabular}
		
		Aufgrund der Wahrheitstabellen ist ersichtlich, dass eine Implikation \textbf{nicht kommutativ} ist.
		\newline
		Darüber hinaus wird die Implikationsumformungsregel $x \Rightarrow y \equiv \lnot y \Rightarrow \lnot x$ verletzt.
		\item
		B ... Backup hat funktioniert
		\newline
		D ... Kein Datenverlust
		
		\begin{tabular}{|l|l|c|c|}
			\firsthline
			B & D & $B \Rightarrow D$ & $\lnot D \Rightarrow \lnot B$ \\			
			\hline
				f & f & w & w \\
				f & w & w & w \\
				w & f & f & f \\
				w & w & w & w \\
			\hline
		\end{tabular}
		
		Wie in der Wahrheitstabelle ersichtlich, sind die Aussagen äquivalent.
		\newline
		Zusätzlich wird die Implikationsumformungsregel $x \Rightarrow y \equiv \lnot y \Rightarrow \lnot x$ erfüllt.
	\end{enumerate}
	
	\section{Aufgaben}
	
	\begin{enumerate}
		\item
		\begin{enumerate}
			\item
			\begin{equation*}
				\underbrace{\lnot (\underbrace{A \land (\underbrace{B \Rightarrow A}_{\text{Implikation}})}_{\text{Konjunktion}})}_{\text{Negation}}
			\end{equation*}
			\begin{tabular}{|l|l|c|c|c|}
				\firsthline
					A & B & B $\Rightarrow$ A & $A \land (B \Rightarrow A)$ & $\lnot (A \land (B \Rightarrow A))$ \\
				\hline
					f & f & w & f & w \\
					f & w & f & f & w \\
					w & f & w & w & f \\
					w & w & w & w & f \\
				\hline
			\end{tabular}
			
			\item
			\begin{equation*}
				\underbrace{A \lor (\underbrace{\underbrace{\lnot B}_{\text{Negation}} \lor C}_{\text{Disjunktion}})}_{\text{Disjunktion}}
			\end{equation*}
			\begin{tabular}{|l|l|l|c|c|c|}
				\firsthline
					A & B & C & $\lnot B$ & $\lnot B \lor C$ & $A \lor (\lnot B \lor C)$ \\
				\hline
					f & f & f & w & w & w \\
					f & f & w & w & w & w \\
					f & w & f & f & f & f \\
					f & w & w & f & w & w \\
					w & f & f & w & w & w \\
					w & f & w & w & w & w \\
					w & w & f & f & f & w \\
					w & w & w & f & w & w \\
				\hline
			\end{tabular}
			
			\item
			\begin{equation*}
				\underbrace{(\underbrace{A \lor D}_{\text{Disjunktion}}) \land \underbrace{\lnot (\underbrace{\underbrace{\lnot B}_{\text{Negation}} \lor C}_{\text{Disjunktion}})}_{\text{Negation}}}_{\text{Konjunktion}}
			\end{equation*}
			\begin{tabular}{|l|l|l|l|c|c|c|c|c|}
				\firsthline
				A & B & C & D & $\lnot$ B & $\lnot B \lor C$ & $\lnot (\lnot B \lor C)$ & $A \lor D$ & $(A \lor D) \land \lnot (\lnot B \lor C)$ \\
				\hline
				f & f & f & f & w & w & f & f & f \\	
				f & f & f & w & w & w & f & f & f \\
				f & f & w & f & w & w & f & f & f \\
				f & w & f & f & f & f & w & f & w \\
				f & f & w & w & w & w & f & w & f \\
				f & w & f & w & f & f & w & w & w \\
				f & w & w & f & f & w & f & f & f \\
				w & f & f & f & w & w & f & w & f \\
				w & f & f & w & w & w & f & w & f \\
				w & f & w & w & w & w & f & w & f \\
				w & w & w & f & f & w & f & w & f \\
				w & f & w & f & w & w & f & w & f \\
				w & w & f & f & f & f & w & w & w \\
				f & w & w & w & f & w & f & w & f \\
				w & w & f & w & f & f & w & w & w \\
				w & w & w & w & f & w & f & w & f \\
				\hline
			\end{tabular}
		\end{enumerate}
		
		\item
		\begin{equation*}
			x = \lnot A \land B
		\end{equation*}
		\begin{tabular}{|l|l|c|c|}
			\firsthline
			A & B & $\lnot A$ & $ \lnot A \land B $ \\
			\hline
			f & f & w & f \\
			f & w & w & w \\
			w & f & f & f \\
			w & w & f & f \\
			\hline
		\end{tabular}
		\newline
		Die Aussage kann nicht wahr sein, da wenn Karl hungrig ist, A den Wert wahr hat und durch die Negation von A das Endergebnis auf einer Konjunktion mit einer falschen Aussage beruht.
		
		\item
		\begin{enumerate}
			\item
			\begin{tabular}{|l|l|c|c|}
				 \firsthline
				 A & B & $B  \Rightarrow A$ & $A \land (B  \Rightarrow A)$ \\
				 \hline
				 f & f & w & f \\
				 f & w & f & f \\
				 w & f & w & w \\
				 w & w & w & w \\
				 \hline
			\end{tabular}
			
			\item
			\begin{tabular}{|l|l|c|c|c|}
				\firsthline
				A & B & $A \land B$ & $\lnot (A \land B)$ & $B \lor \lnot (A \land B)$ \\
				\hline
				f & f & f & w & w \\
				f & w & f & w & w \\
				w & f & f & w & w \\
				w & w & w & f & w \\
				\hline
			\end{tabular}
			\newline
			Es handelt sich um eine Tautologie.
			
			\item
			\begin{tabular}{|l|l|c|c|c|}
				\firsthline
				A & B & $\lnot$ B & $A \land \lnot B$ & $B \Rightarrow (A \land \lnot B)$ \\
				\hline
				f & f & w & f & w \\
				f & w & f & f & f \\
				w & f & w & w & w \\
				w & w & f & f & f \\
				\hline
			\end{tabular}
		\end{enumerate}
		
		\item
		\begin{tabular}{|l|l|l|c|c|c|c|c|c|}
			\firsthline
			x & y & z & $\lnot$ y & $x \land (\lnot y)$ & $\lnot (x \land (\lnot y))$ & $x \lor z$ & $y \land (x \lor z)$ & $(\lnot (x \land (\lnot y))) \lor (y \land (x \lor z))$ \\
			\hline
			f & f & f & w & f & w & f & f & \textbf{w} \\
			f & f & w & w & f & w & w & f & \textbf{w} \\
			f & w & f & f & f & w & f & f & \textbf{w} \\
			f & w & w & f & f & w & w & w & \textbf{w} \\
			w & f & f & w & w & f & w & f & \textbf{f} \\
			w & f & w & w & w & f & w & f & \textbf{f} \\
			w & w & f & f & f & w & w & w & \textbf{w} \\
			w & w & w & f & f & w & w & w & \textbf{w} \\
			\hline
		\end{tabular}
		
		\begin{tabular}{|l|l|c|c|}
			\firsthline
			x & y & $\lnot$ x & $\lnot x \lor y$ \\
			\hline
			f & f & w & \textbf{w} \\
			f & w & w & \textbf{w} \\
			w & f & f & \textbf{f} \\
			w & w & f & \textbf{w} \\
			\hline
		\end{tabular}
		
		\item
		\begin{enumerate}
			\item
			\begin{equation*}
				\text{Anzahl } i = 2^{n} = 2^{2} = 4
			\end{equation*}
			
			\item
			\begin{equation*}
				i = 2^{n} = 2^{4} = 16
			\end{equation*}
			
			\item
			\begin{equation*}
				i = 2^{n}
			\end{equation*}
		\end{enumerate}
		
		\item
		\begin{enumerate}
			\item
			\begin{equation*}
				(((x \lor \lnot y) \land (x \lor y)) \land (\lnot x \lor \lnot y)) \lor y
			\end{equation*}
			\begin{equation*}
				(((x \lor \lnot y) \land (x \lor y)) \lor y) \land ((\lnot x \lor \lnot y)) \lor y)
			\end{equation*}
			\begin{equation*}
				(((x \lor \lnot y) \lor y) \land (y \lor (x \lor y))) \land ((\lnot x \lor \lnot y)) \lor y)
			\end{equation*}
			\begin{equation*}
				((x \lor (\lnot y \lor y)) \land ((y \lor y) \lor x)) \land ((\lnot x \lor \lnot y)) \lor y)
			\end{equation*}
			\begin{equation*}
				((x \lor w) \land (y \lor x)) \land ((\lnot x \lor \lnot y)) \lor y)
			\end{equation*}
			\begin{equation*}
				(w \land (y \lor x)) \land ((\lnot x \lor \lnot y)) \lor y)
			\end{equation*}
			\begin{equation*}
				(y \lor x) \land ((\lnot x \lor \lnot y)) \lor y)
			\end{equation*}
			\begin{equation*}
				((y \lor x) \land (\lnot x \lor \lnot y)) \lor ((y \lor x) \land y)
			\end{equation*}
			\begin{equation*}
				((y \lor x) \land (\lnot x \lor \lnot y)) \lor ((y \lor x) \land y)
			\end{equation*}
			\begin{equation*}
				((y \lor x) \land (\lnot x \lor \lnot y)) \lor y
			\end{equation*}
			\begin{equation*}
				((y \lor x) \lor y) \land (y \lor (\lnot x \lor \lnot y))
			\end{equation*}
			\begin{equation*}
				(x \lor (y \lor y)) \land (\lnot x \lor (y \lor \lnot y))
			\end{equation*}
			\begin{equation*}
				(x \lor y) \land (\lnot x \lor w)
			\end{equation*}
			\begin{equation*}
				(x \lor y) \land w
			\end{equation*}
			\begin{equation*}
				(x \lor y)
			\end{equation*}
			
			\item
			\begin{equation*}
				(\lnot y \land \lnot x) \land ((y \lor \lnot x) \land (y \lor \lnot y))
			\end{equation*}
			\begin{equation*}
				(\lnot y \land \lnot x) \land ((y \lor \lnot x) \land w)
			\end{equation*}
			\begin{equation*}
				(\lnot y \land \lnot x) \land (y \lor \lnot x)
			\end{equation*}
			\begin{equation*}
				(\lnot y) \land (\lnot x \land (y \lor \lnot x))
			\end{equation*}
			\begin{equation*}
				(\lnot y) \land (\lnot x \land (y \lor \lnot x))
			\end{equation*}
			\begin{equation*}
				(\lnot y) \land (\lnot x)
			\end{equation*}
			
			\item
			\begin{equation*}
				\lnot (x \lor y) \land (y \lor (\lnot x \land \lnot y))
			\end{equation*}
			\begin{center}
				Siehe Beispiel b.
			\end{center}
			
			\item
			\begin{equation*}
				(\lnot x \land \lnot y) \lor ((x \lor x) \land (x \lor y))
			\end{equation*}
			\begin{equation*}
				(\lnot x \land \lnot y) \lor (x \land (x \lor y))
			\end{equation*}
			\begin{equation*}
				((\lnot x \land \lnot y) \lor x) \land ((\lnot x \land \lnot y) \lor (x \lor y))
			\end{equation*}
			\begin{equation*}
				((\lnot x \lor x) \land (x \lor \lnot y)) \land (((\lnot x \land \lnot y) \lor x) \lor y)
			\end{equation*}
			\begin{equation*}
				(w \land (x \lor \lnot y)) \land (((\lnot x \land \lnot y) \lor x) \lor y)
			\end{equation*}
			\begin{equation*}
				(x \lor \lnot y) \land (((\lnot x \lor x) \land (x \lor \lnot y)) \lor y)
			\end{equation*}
			\begin{equation*}
				(x \lor \lnot y) \land ((w \land (x \lor \lnot y)) \lor y)
			\end{equation*}
			\begin{equation*}
				(x \lor \lnot y) \land ((x \lor \lnot y) \lor y)
			\end{equation*}
			\begin{equation*}
				(x \lor \lnot y) \land (x \lor (\lnot y \lor y))
			\end{equation*}
			\begin{equation*}
				(x \lor \lnot y) \land (x \lor w)
			\end{equation*}
			\begin{equation*}
				(x \lor \lnot y) \land w
			\end{equation*}
			\begin{equation*}
				(x \lor \lnot y)
			\end{equation*}
			
			\item
			\begin{equation*}
				((x \lor \lnot x) \land (x \lor y)) \land ((x \lor x) \land (x \lor y))
			\end{equation*}
			\begin{equation*}
				(w \land (x \lor y)) \land (x \land (x \lor y))
			\end{equation*}
			\begin{equation*}
				(x \lor y) \land (x)
			\end{equation*}
			\begin{equation*}
				x
			\end{equation*}
		\end{enumerate}
		
		\item
		\begin{enumerate}
			\item
			\begin{tabular}{|l|l|c|c|c|c|c|}
				\firsthline
					A & B & $\lnot B$ & $A \land (\lnot B)$ & $\lnot A$ & $B \land (\lnot A)$ & $(A \land (\lnot B)) \lor (B \land (\lnot A))$ \\
				\hline
					f & f & w & f & w & f & f \\
					f & w & f & f & w & w & w \\
					w & f & w & w & f & f & w \\
					w & w & f & f & f & f & f \\
				\hline
			\end{tabular}
			
			\item
			\begin{tabular}{|l|l|c|c|c|}
				\firsthline
				A & B & $A \Rightarrow B$ & $B \Rightarrow A$ & $(A \Rightarrow B) \land (B \Rightarrow A)$ \\
				\hline
				f & f & w & w & w \\
				f & w & w & f & f \\
				w & f & f & w & f \\
				w & w & w & w & w \\
				\hline
			\end{tabular}
		\end{enumerate}
		
		\item
		\begin{tabular}{|l|l|c|}
			\firsthline
			x & y & $\lnot (x \otimes y)$ \\
			\hline
			f & f & t \\
			f & t & f \\
			t & f & f \\
			t & t & t \\
			\hline
		\end{tabular}
		
		\item
		"Wenn die Sonne scheint, ist Lukas glücklich. Wenn Lukas glücklich ist, funktioniert das Backup.". Ist im ersten Punkt die Prämisse wahr, ist die Konklusion ebenfalls wahr. Da die Konklusion der ersten Implikation ebenfalls als Prämisse der zweiten Implikation dient, ist deren Konklusion ebenfalls wahr. Das heißt, das Backup hat funktioniert.
	\end{enumerate}
\end{document}
