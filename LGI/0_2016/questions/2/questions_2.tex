\documentclass[a4paper]{article}
\usepackage{ucs}
\usepackage[utf8x]{inputenc}
\usepackage[T1]{fontenc}
\usepackage{german}
\usepackage{a4,ngerman}
\usepackage[ngerman]{babel}
\usepackage{graphicx}
\usepackage[]{cite}
\usepackage{fancyhdr}
\usepackage{amsmath}
\pagestyle{fancy}
\selectlanguage{german}
\usepackage{array}
\usepackage{mathtools}
\usepackage{amssymb}
\cfoot{\textcopyright Lukas Sch�rghuber, S1610307103}

\begin{document}
	\begin{enumerate}
		\item
		Die Reihenfolge der Quantoren ist wichtig f�r die Reihenfolge der Bindung der Variablen.
		Folgt z.B. auf einen Existenzquantor ein Allquantor, so ist der vom Existenzquantor gebundene Wert f�r jeden Wert, den die durch den Allquantor gebundene Variable annehmen kann, fix.
		Werden die Quantoren in umgekehrter Reihenfolge geschrieben, kann die durch den Existenzquantor gebundene Variable jeden passenden Wert zu der vom Allquantor gebundenen Variable annehmen.
		
		\item
		Es ist wichtig, x auf einen beliebigen Wert zu setzen, da der Allquantor genau das fordert. Kann man die Aussage f�r einen beliebigen Wert beweisen, kann jeden Wert f�r den beliebigen Wert verwenden.
		Dar�ber hinaus ist es wichtig x auf einen fixen Wert zu setzen, da ansonsten die Aussage so modifiziert werden kann, damit sie nach Belieben wahr oder falsch ist.
		
		\item
		???
		
		\item
		Eine Problemspezifikation ist eine mathematische Beschreibung der Abstraktion eines Problems.
		Sie gibt an in welchem Zusammenhang gegebene und gesuchte Gr��en stehen m�ssen, um die L�sung des Problems darzustellen.
		Sie besteht aus den folgenden Teilen:
		\begin{itemize}
			\item Gegebene Gr��en
			\item Gesuchte Gr��en
			\item Typisierung der gegebenen und gesuchten Gr��en
			\item Zusammenhang zwischen den gegebenen und gesuchten Gr��en
			\item Eindeutige Definition der erlaubten Funktions- und Pr�dikatskonstanten
		\end{itemize}
	\end{enumerate}
\end{document}