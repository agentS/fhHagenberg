\documentclass[a4paper]{article}
\usepackage{ucs}
\usepackage[utf8x]{inputenc}
\usepackage[T1]{fontenc}
\usepackage{german}
\usepackage{a4,ngerman}
\usepackage[ngerman]{babel}
\usepackage{graphicx}
\usepackage[]{cite}
\usepackage{fancyhdr}
\usepackage{amsmath}
\pagestyle{fancy}
\selectlanguage{german}
\usepackage{array}
\usepackage{mathtools}
\usepackage{amssymb}
\cfoot{\textcopyright Lukas Schörghuber, S1610307103}

\begin{document}
	\begin{enumerate}
		\item
		\begin{enumerate}
			\item
			Es gibt mindestens einen Wert x, der jeden Wert y ohne Rest teilt.
			\newline
			Diese Aussage ist wahr, da jede ganze Zahl durch 1 und durch sich selbst geteilt wird und x diese Werte annehmen kann. Diese Aussage ist ebenfalls für die ganzen Zahlen wahr, da 1 alle ganzen Zahlen teilt.
			
			\begin{equation*}
				x | y \leftarrow x = 1
			\end{equation*}
			\begin{equation*}
				1 | y \equiv w
			\end{equation*}
			
			\item
			Für alle x gibt es mindestens ein y, welches von allen x ohne Rest geteilt wird.
			\newline
			Diese Aussage ist für die natürlichen Zahlen wahr, da x vor y gebunden wird und y somit immer den Wert von y annehmen kann. Für die ganzen Zahlen ist die Aussage allerdings falsch, da x den Wert 0 annehmen kann und eine Division durch 0 nicht möglich ist.
			
			\item
			Es gibt mindestens ein x, dass von allen y geteilt wird.
			\newline
			In den natürlichen Zahlen ist diese Aussage falsch, da keine Zahl durch alle beliebigen Zahlen geteilt werden kann. In den ganzen Zahlen ist diese Aussage allerdings richtig, da 0 von allen Zahlen ohne Rest geteilt wird.
			
			\item
			Für alle Werte x gibt es mindestens ein Element y, dass von den x ohne Rest geteilt wird.
			\newline
			Diese Aussage ist wahr, da y den Wert von 1 annehmen kann und jede Zahl durch 1 ohne Rest geteilt wird. Darüber hinaus wird x vor y gebunden und y kann den Wert von x annehmen.
			\begin{equation*}
				y | x \leftarrow y = 1
			\end{equation*}
			\begin{equation*}
				1 | x \equiv w
			\end{equation*}
			\begin{equation*}
				y | x \leftarrow y = x
			\end{equation*}
			\begin{equation*}
				x | x \equiv w
			\end{equation*}
		\end{enumerate}
		
		\item
		\begin{enumerate}
			\item
			\begin{equation*}
				P(x, y) :\Leftrightarrow x^{2} + y^{2} > 10
			\end{equation*}
			
			\item
			\begin{equation*}
				Q(x) :\Leftrightarrow \exists a, b \text{ } a \neq b \land \text{isPrime}(a) \land \text{isPrime}(b) \land x = a^{2} + b^{2}
			\end{equation*}
		\end{enumerate}
		
		\item
		\begin{enumerate}
			\item
			\begin{equation*}
				\forall x \text{ } x \geq 1 \land x < 40 \Rightarrow \text{prim}(x^{2} + x + 41)
			\end{equation*}
			
			\item
			\begin{equation*}
				\forall p \text{ } p \geq 2 \Rightarrow \exists n \text{ } n \geq 2 \land \text{prim}(p) \land p \leq n
			\end{equation*}
		\end{enumerate}
		
		\item
		\begin{enumerate}
			\item Die FK "=" erhält zwei Wahrheitswerte (die Ergebnisse der PK "prim") als Parameter.
			\item Die PK "prim" erhält einen Wahrheitswert als Parameter.
			\item Die PK "=" erhält einen Wahrheitswert (Ergebnis der PK "prim") als Parameter.
			\item Diese Aussage ist nicht gültig, wenn der Wert von x und y vertauscht wird. Der Absolutbetrag der Differenz muss 2 ergeben.
			\newline			
			\textbf{Alternative:} Es kann an den Term angehängt werden (mittels Konjunktion), dass x größer als y ist.
		\end{enumerate}
		
		\item
		\begin{equation*}
			\lnot (\forall p \text{ } p \geq 2 \Rightarrow \exists n \text{ } n \geq 2 \land \text{prim}(p) \land p \leq n)
		\end{equation*}
		\begin{equation*}
			\exists p \text{ } \lnot(p \geq 2 \Rightarrow \exists n \text{ } n \geq 2 \land \text{prim}(p) \land p \leq n)
		\end{equation*}
		\begin{equation*}
			\exists p \text{ } \lnot(\lnot (p \geq 2) \lor (\exists n \text{ } n \geq 2 \land \text{prim}(p) \land p \leq n))
		\end{equation*}
		\begin{equation*}
			\exists p \text{ } ((p \geq 2) \land \lnot (\exists n \text{ } n \geq 2 \land \text{prim}(p) \land p \leq n))
		\end{equation*}
		\begin{equation*}
			\exists p \text{ } ((p \geq 2) \land \forall n \text{ } \lnot (n \geq 2 \land (\text{prim}(p) \land p \leq n)))
		\end{equation*}
		\begin{equation*}
			\exists p \text{ } p \geq 2 \land \forall n \text{ } \lnot (n \geq 2) \lor \lnot (\text{prim}(p) \land p \leq n)
		\end{equation*}
		\textbf{Lösung A}
		\begin{equation*}
			\exists p \text{ } p \geq 2 \land \forall n \text{ } \lnot (n \geq 2) \lor \lnot (\text{prim}(p)) \lor \lnot (p \leq n)
		\end{equation*}
		\textbf{Lösung B}
		\begin{equation*}
			\exists p \text{ } p \geq 2 \land \forall n \text{ } n \geq 2 \Rightarrow \lnot (\text{prim}(p) \land (p \geq n))
		\end{equation*}
		\begin{equation*}
			\exists_{p \geq 2} \forall_{n \geq 2} \lnot(\text{prim}(p) \land p \leq n)
		\end{equation*}
		
		\item
		Es gilt zu beweisen, dass es unendlich oft vorkommt, dass $p + 2$ keine Primzahl ist, wenn $p$ eine Primzahl ist. Beispiele dafür sind 7 und 9, 11 und 13 sowie 23 und 25. Zu beachten ist, dass all die erwähnten Zahlen ungerade sind.
		\newline
		Um den Widerspruch zu bilden, nimmt man an, dass dies nicht so ist und ab einer bestimmten Primzahl $q$ alle Zahlen, die durch $q + 2$ gebildet werden, ebenfalls Primzahlen sind. Daraus ergibt sich, dass $(q + 2) + 2$ ebenfalls eine Primzahl ist. Daraus folgt, dass jede ungerade Zahl ab $q$ eine Primzahl ist.
		\newline
		Nun kann man das Produkt $Q$ folgendermaßen bilden: $Q = q \cdot (q + 2)$. Da sowohl $q$ als auch $q + 2$ ungerade sind, ist auch das Ergebnis deren Multiplikation ungerade und $Q$ ist größer als $q$: $Q > q$.
		\newline
		Folglich müsste auch $Q$ eine Primzahl sein, da sie ja ungerade und größer als q ist.
		Allerdings steht das im Widerspruch zu der Tatsache, dass $Q$ das Ergebnis einer Multiplikation ist und somit durch $q$ und $q + 2$ ohne Rest geteilt werden kann.
		\newline
		Somit ist das Gegenteil des Widerspruches bewiesen.
	\end{enumerate}
\end{document}