\documentclass[a4paper]{article}
\usepackage{ucs}
\usepackage[utf8x]{inputenc}
\usepackage[T1]{fontenc}
\usepackage{german}
\usepackage{a4,ngerman}
\usepackage[ngerman]{babel}
\usepackage{graphicx}
\usepackage[]{cite}
\usepackage{fancyhdr}
\usepackage{amsmath}
\pagestyle{fancy}
\selectlanguage{german}
\usepackage{array}
\usepackage{mathtools}
\usepackage{amssymb}
\cfoot{\textcopyright Lukas Schörghuber, S1610307103}

\begin{document}
	\begin{enumerate}
		\item
		\begin{enumerate}
			\item
			Es gibt mindestens einen Wert x, der jeden Wert y ohne Rest teilt.
			\newline
			Diese Aussage ist wahr, da jede ganze Zahl durch 1 und durch sich selbst geteilt wird und x diese Werte annehmen kann.
			
			\begin{equation*}
				x | y \leftarrow x = 1
			\end{equation*}
			\begin{equation*}
				1 | y \equiv w
			\end{equation*}
			\begin{equation*}
				x | y \leftarrow x = y
			\end{equation*}
			\begin{equation*}
				y | y \equiv w
			\end{equation*}
			
			\item
			Für alle x gibt es mindestens ein y, welches von allen x ohne Rest geteilt wird.
			\newline
			Diese Aussage ist falsch, da y vor x gebunden wird und somit für jeden beliebigen Wert von x den selben Wert annehmen muss und nur die Zahl 0 von allen Zahlen ohne Rest geteilt wird. Allerdings kann y aufgrund der Beschränkung des Universums auf die ganzen Zahlen nicht den Wert 0 annehmen.
			
			\item
			Es gibt mindestens ein x gibt es eine beliebige Menge von y, die x ohne Rest teilen.
			\newline
			
			
			\item
			Für alle Werte x gibt es mindestens ein Element y, dass von den x ohne Rest geteilt wird.
			\newline
			Diese Aussage ist wahr, da y den Wert von 1 annehmen kann und jede Zahl durch 1 ohne Rest geteilt wird.
			\begin{equation*}
				y | x \leftarrow y = 1
			\end{equation*}
			\begin{equation*}
				1 | x \equiv w
			\end{equation*}
		\end{enumerate}
		
		\item
		\begin{enumerate}
			\item
			\begin{equation*}
				P(x, y) :\Leftrightarrow x^{2} + y^{2} > 10
			\end{equation*}
			
			\item
			\begin{equation*}
				Q(x) :\Leftrightarrow \exists a, b \text{ } a \neq b \land \text{isPrime}(a) \land \text{isPrime}(b) \land x = a^{2} + b^{2}
			\end{equation*}
		\end{enumerate}
		
		\item
		\begin{enumerate}
			\item
			\begin{equation*}
				\forall x \text{ } x \geq 1 \land x < 40 \Rightarrow \text{prim}(x^{2} + x + 41)
			\end{equation*}
			
			\item
			\begin{equation*}
				\forall p \text{ } p \geq 2 \Rightarrow \exists n \text{ } n \geq 2 \land \text{prim}(p) \land p \leq n
			\end{equation*}
		\end{enumerate}
		
		\item
		\begin{enumerate}
			\item Die FK "=" erhält zwei Wahrheitswerte (die Ergebnisse der PK "prim") als Parameter.
			\item Die PK "prim" erhält einen Wahrheitswert als Parameter.
			\item Die PK "=" erhält einen Wahrheitswert (Ergebnis der PK "prim") als Parameter.
			\item Diese Aussage ist nicht gültig, wenn der Wert von x und y vertauscht wird. Der Absolutbetrag der Differenz muss 2 ergeben.
		\end{enumerate}
		
		\item
		\begin{equation*}
			\lnot (\forall p \text{ } p \geq 2 \Rightarrow \exists n \text{ } n \geq 2 \land \text{prim}(p) \land p \leq n)
		\end{equation*}
		\begin{equation*}
			\exists p \text{ } \lnot(p \geq 2 \Rightarrow \exists n \text{ } n \geq 2 \land \text{prim}(p) \land p \leq n)
		\end{equation*}
		\begin{equation*}
			\exists p \text{ } \lnot(\lnot (p \geq 2) \lor (\exists n \text{ } n \geq 2 \land \text{prim}(p) \land p \leq n))
		\end{equation*}
		\begin{equation*}
			\exists p \text{ } ((p \geq 2) \land \lnot (\exists n \text{ } n \geq 2 \land \text{prim}(p) \land p \leq n))
		\end{equation*}
		\begin{equation*}
			\exists p \text{ } ((p \geq 2) \land \forall n \text{ } \lnot (n \geq 2 \land (\text{prim}(p) \land p \leq n)))
		\end{equation*}
		\begin{equation*}
			\exists p \text{ } p \geq 2 \land \forall n \text{ } \lnot (n \geq 2) \lor \lnot (\text{prim}(p) \land p \leq n)
		\end{equation*}
		\begin{equation*}
			\exists p \text{ } p \geq 2 \land \forall n \text{ } \lnot (n \geq 2) \lor \lnot (\text{prim}(p)) \lor \lnot (p \leq n)
		\end{equation*}
	\end{enumerate}
\end{document}