\documentclass[a4paper]{article}
\usepackage{ucs}
\usepackage[utf8x]{inputenc}
\usepackage[T1]{fontenc}
\usepackage{german}
\usepackage{a4,ngerman}
\usepackage[ngerman]{babel}
\usepackage{graphicx}
\usepackage[]{cite}
\usepackage{fancyhdr}
\usepackage{amsmath}
\pagestyle{fancy}
\selectlanguage{german}
\usepackage{array}
\usepackage{mathtools}
\cfoot{\textcopyright Lukas Schörghuber, S1610307103}

\begin{document}
	\section{Mögliche Junktoren mit n Stellen}
	
	\begin{center}
		\begin{tabular}{|c|r|}
			\firsthline
				atomare Aussagen (Stellen) & n \\
			\hline
				Wahrheitswerte & w \\
			\hline
				mögliche Kombinationen & k \\
			\hline
		\end{tabular}
	\end{center}
	
	\begin{equation*}
		\text{mögliche Junktoren} = n^{m^{k}}
	\end{equation*}
	\begin{equation*}
		k = m^{n}
	\end{equation*}
	\begin{equation*}
		\text{mögliche Junktoren} = n^{m^{m^{n}}}
	\end{equation*}
	
	\section{Beweis modus ponens}
	\begin{equation*}
		x \Rightarrow y \equiv w \text{ , } x \equiv w 
	\end{equation*}
	
	\begin{center}
		\begin{tabular}{|l|l|c|}
			\firsthline
				x & y & $x \Rightarrow y$ \\
			\hline
				f & f & w \\
				f & w & w \\
				\textbf{w} & f & f \\
				\textbf{w} & \textbf{w} & \textbf{w} \\
			\hline
		\end{tabular}
	\end{center}
	
	Zuerst werden die Zeilen betrachtet, in denen x wahr ist. Danach wird die Schnittmenge mit der Zeile, in der $x \Rightarrow y$ wahr ist gebildet. Das Ergebnis ist nur die Zeile, in der y ebenfalls wahr ist.
	
	\section{If-Then-Else-Junktor}
	Gesucht ist ein Junktor, der nach dem folgenden Prinzip funktioniert: if c then x else y
	\newline
	Der Junktor liefert den Wahrheitswert von y zurück, wenn c falsch ist und falls c wahr ist, wird der Wahrheitswert von x geliefert.
	
	\begin{center}
		\begin{tabular}{|l|l|l|c|}
			\firsthline
				c & x & y & ??? \\
			\hline
				f & f & \textbf{f} & \textbf{f} \\
				f & f & \textbf{w} & \textbf{w} \\
				f & w & \textbf{f} & \textbf{f} \\
				f & w & \textbf{w} & \textbf{w} \\
				w & \textbf{f} & f & \textbf{f} \\
				w & \textbf{f} & w & \textbf{f} \\
				w & \textbf{w} & f & \textbf{w} \\
				w & \textbf{w} & w & \textbf{w} \\
			\hline
		\end{tabular}
	\end{center}
	
	\begin{equation*}
		(\lnot c \land \lnot x \land y) \lor (\lnot c \land x \land y) \lor (c \land x \land \lnot y) \lor (c \land x \land y)
	\end{equation*}
	\begin{equation*}
		\equiv (c \land x) \lor (\lnot c \land y)
	\end{equation*}
\end{document}