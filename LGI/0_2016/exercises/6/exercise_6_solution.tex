\documentclass[a4paper]{article}
\usepackage{ucs}
\usepackage[utf8x]{inputenc}
\usepackage[T1]{fontenc}
\usepackage{german}
\usepackage{a4,ngerman}
\usepackage[ngerman]{babel}
\usepackage{graphicx}
\usepackage[]{cite}
\usepackage{fancyhdr}
\usepackage{amsmath}
\pagestyle{fancy}
\selectlanguage{german}
\usepackage{array}
\usepackage{mathtools}
\usepackage{amssymb}
\cfoot{\textcopyright Lukas Schörghuber, S1610307103}

\begin{document}
	\begin{enumerate}
		\item
		\begin{enumerate}
			\item
			\begin{equation*}
				\bigcup(M) := \{ a | \exists m \text{ } m \in M \land a \in m \}
			\end{equation*}
			
			\item
			\begin{equation*}
				\bigcap(M) := \{ a | \exists m \text{ } m \in M \land a \in m \land \forall l \text{ } l \in M \land m \neq l  \Rightarrow a \in l \}
			\end{equation*}
		\end{enumerate}
		
		\item
		\begin{enumerate}
			\item
			\begin{figure}[ht!]
				\begin{center}
					\includegraphics[height=45mm]{1a.eps}
					\caption{gerichteter Graph für Aufgabe 2.}
				\end{center}
			\end{figure}
			
			\item
			\begin{enumerate}
				\item
				\begin{equation*}
					M := \{ (5, 5) \}
				\end{equation*}
				
				\item
				\begin{equation*}
					M := \{ (3, 1), (3, 2) \}
				\end{equation*}
			\end{enumerate}
			
			\item
			\begin{itemize}
				\item \textbf{reflexiv}: \textbf{falsch}, da die Knoten 1, 3 und 4 über keine Selbstpfeile verfügen
				\item \textbf{irreflexiv}: \textbf{falsch}, da die Knoten 2 und 5 über Selbstpfeile verfügen
				\item \textbf{symmetrisch}: \textbf{falsch}, da nur zwischen den Knoten 1 und 3 Doppelpfeile vorhanden sind
				\item \textbf{asymmetrisch}: \textbf{falsch}, da zwischen Knoten 1 und 3 Doppelpfeile vorhanden sind und die Knoten 2 und 5 über Selbstpfeile verfügen
				\item \textbf{antisymmetrisch}: \textbf{falsch}, da zwischen Knoten 1 und 3 Doppelpfeile vorhanden sind
				\item \textbf{transitiv}: \textbf{wahr}
			\end{itemize}
		\end{enumerate}
		
		\item
		Zur Ermittlung der Ergebnisse muss in die Definition eingesetzt werden.
		\newline
		\begin{tabular}{|c|c|c|c|c|c|c|}
			\firsthline
				Relation & reflexiv & irreflexiv & symmetrisch & asymmetrisch & antisymmetrisch & transitiv \\
			\hline
				$\in$ & falsch & wahr & falsch & wahr & wahr & falsch \\
			\hline
				$\subseteq$ & wahr & falsch & falsch & falsch & wahr & falsch \\
			\hline
		\end{tabular}
		
		\clearpage
		\item
		Universum: $\mathbb{N}$
		\begin{itemize}
			\item \textbf{relfexiv}: \textbf{wahr}, da jede Zahl sich selbst ohne Rest teilt
			\item \textbf{irreflexiv}: \textbf{falsch}, da jede Zahl sich selbst ohne Rest teilt
			\item \textbf{symmetrisch}: \textbf{falsch}, da, falls y größer als x ist, y von x ohne Rest geteilt wird, allerdings x von y immer mit Rest geteilt wird, da das Ergebnis der Division 0 ist.
			\item \textbf{asymmetrisch}: \textbf{falsch}, da x sich selbst ohne Rest teilt
			\item \textbf{antisymmetrisch}: \textbf{wahr}, da die Relation nicht symmetrisch ist und x nur sich selbst ohne Rest teilt, wenn diese gleich ist
			\item \textbf{transitiv}: \textbf{wahr}, da, falls x y ohne Rest teilt und y z ohne Rest teilt, x z ohne Rest teilt. Z.B.: $3 | 9 \land 9 | 81 \Rightarrow 3 | 81$
		\end{itemize}
		
		\item
		Universum: $\mathbb{Z}$
		\begin{enumerate}
			\item
			\begin{equation*}
				R_{0} := \{(1, 1), (1, 2), (2, 2), (2, 3), (3, 4)\}
			\end{equation*}
			
			\item
			\begin{equation*}
				R_{1} := \{(-2, 3), (-3, 4), (-1, 2), (2, -1), (3, -2)\}
			\end{equation*}
		\end{enumerate}
		
		\item
		\begin{equation*}
			R := \{(1, 1), (1, 3), (2, 1), (2, 2), (3, 3), (3, 4), (4, 3), (4, 4)\}
		\end{equation*}	
		
		\begin{figure}[ht!]
			\begin{center}
				\includegraphics[height=45mm]{6.eps}
				\caption{gerichteter Graph für Aufgabe 6.}
			\end{center}
		\end{figure}
		
	\end{enumerate}
\end{document}
