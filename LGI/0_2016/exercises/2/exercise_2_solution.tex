\documentclass[a4paper]{article}
\usepackage{ucs}
\usepackage[utf8x]{inputenc}
\usepackage[T1]{fontenc}
\usepackage{german}
\usepackage{a4,ngerman}
\usepackage[ngerman]{babel}
\usepackage{graphicx}
\usepackage[]{cite}
\usepackage{fancyhdr}
\usepackage{amsmath}
\pagestyle{fancy}
\selectlanguage{german}
\usepackage{array}
\usepackage{mathtools}
\usepackage{amssymb}
\cfoot{\textcopyright Lukas Schörghuber, S1610307103}

\begin{document}
	\begin{enumerate}
		\item
		\begin{enumerate}
			\item
			Es gibt mindestens ein Element für x und mindestens ein Element für y für die gilt, dass x addiert mit y 0 ergibt.
			\newline
			Die Aussage ist wahr, da für y durch den Wert $-x$ substituiert werden kann und somit das Ergebnis immer 0 ist.
			\begin{equation*}
				x + (-x) = 0
			\end{equation*}
			\begin{equation*}
				\equiv x - x = 0
			\end{equation*}
			Man kann z.B. 2 in die obere Aussage einsetzen:
			\begin{equation*}
				2 - 2 = 0
			\end{equation*}
			
			\item
			Für alle x gilt, dass es mindestens ein y gibt, für das gilt, dass die Addition von x mit y 0 ergibt.
			\newline
			Diese Aussage ist falsch, da y erst nach x quantisiert wird und somit ein y bereits fix ist, während der Wert von x noch beliebig sein kann und somit ein y addiert mit allen möglichen Belegungen für x 0 ergeben muss.
			
			\begin{equation*}
				x + y = 0
			\end{equation*}
			Man kann z.B. 5 für x und -5 für y in die obere Aussage einsetzen:
			\begin{equation*}
				5 - 5 = 0
			\end{equation*}
			Setzt man nun den Wert 10 für x ein und verändert y nicht, was ja durch die Bindung vorausgesetzt wird, ist das Ergebnis nicht mehr 0:
			\begin{equation*}
				10 - 5 = 5 \neq 0
			\end{equation*}
			
			\item
			Es gibt mindestens ein Element für x und beliebig viele Elemente für y, für die gilt, dass deren Addition den Wert 0 ergibt.
			\newline
			Diese Aussage ist wahr, da x durch -y substituiert werden kann und das x in der Aussage genau den Wert jedes y annehmen kann, da y zuerst gebunden wird.
			
			\begin{equation*}
				(-y) + y = 0
			\end{equation*}
			\begin{equation*}
				\equiv y - y = 0
			\end{equation*}
			Man kann z.B. 1180 in die obere Aussage einsetzen:
			\begin{equation*}
				1180 - 1180 = 0
			\end{equation*}
			
			\item
			Für jede Variablenbelegung von x gibt es beliebig viele Werte, die y annehmen kann, deren Addition den Wert 0 ergibt.
			\newline
			Diese Aussage ist falsch, da sowohl x und y beliebige Werte annehmen können und x jetzt nicht mehr durch den äquivalenten negierten Wert ersetzt werden kann und umgekehrt.
			\newline
			So ist beispielsweise die Aussage $42 - 42 = 0$ für $x = 42$ und $y = 42$ wahr, jedoch nicht mehr für $x = 42$ und $y = 1180$: $42 - 1180 \neq 0$.
		\end{enumerate}
		
		\item
		\begin{enumerate}
			\item
			\begin{equation*}
				\exists x, y \text{ istMensch}(x) \land \text{ istMensch}(y) \land \text{istBruder}(x, y)
			\end{equation*}
			
			\item
			\begin{equation*}
				\forall x \text{ istStudentin}(x) \land \text{studienort}(x, \text{'Hagenberg'}) \Rightarrow \text{interesse}(x, \text{'Informatik'})
			\end{equation*}
			
			\item
			\begin{equation*}
				\forall x \text{ (istMensch}(x) \land \exists y \text{ istMensch}(y)) \Rightarrow \text{istVater}(y, x)
			\end{equation*}
			
			\item
			\begin{equation*}
				\forall x \text{ istMensch}(x) \land \exists y \forall z \land y \neq z \land \text{ istMensch}(y) \land \text{ istMensch}(z) \Rightarrow \lnot \text{ istVater(z, x)}
			\end{equation*}
		\end{enumerate}
		
		\item
		\begin{enumerate}
			\item
			\begin{equation*}
				\begin{aligned}
					\exists x, y \text{ istMensch}(x) \land \text{ istMensch}(y) \land \text{ studiert}(x, \text{ 'MBI'}) \land \text{ studiert}(y, \text{ 'MBI'}) \\ \land \\ (x \neq y) \land \text{interesse}(x, \text{'Mathematik'}) \land \text{interesse}(y, \text{'Mathematik'})
				\end{aligned}
			\end{equation*}
			
			\item
			\begin{equation*}
				\lnot (\exists x, \forall y \text{ is}\mathbb{N}(x) \land \text{ is}\mathbb{N}(y) \land x > y)
			\end{equation*}
			
			\item
			\begin{equation*}
				\forall y \exists x \text{ istMensch}(x) \land \text{ istName}(\text{'Klaus'}, x) \land \text{ istFilm}(y) \land \text{ Kinobesuch}(x) \Rightarrow \text{gefällt}(y, x)
			\end{equation*}
		\end{enumerate}
		
		\item
		\begin{enumerate}
			\item
			Diese Aussage impliziert, dass alle Katzen graue Katzen sind.
			
			\item
			Die PK '=' erhält auf ihrer linken Seite einen Wahrheitswert, nämlich das Ergebnis von 'istKatze', als Eingabeparameter.
			
			\item
			Es ist nicht möglich, dass die PK '=' die Wahrheitswerte der PK 'istKatze' und 'istGrau' als Eingabeparameter erhält.
			
			\item
			Es ist nicht möglich, dass die Prädikatkonstante 'istKatze' den Wahrheitswert der PK 'istGrau' als Eingabeparameter erhält.
			
			\item
			Es ist nicht möglich, dass die PK 'istKatze' den Wahrheitswert der PK '=' als Eingabeparameter erhält.
			
			\item
			Diese Aussage ist aufgrund der Implikation ebenfalls wahr, wenn ein beliebiger Gegenstand, der keine Katze ist, grau ist.
		\end{enumerate}
		
		\item
		\begin{enumerate}
			\item
			Auf allen Felder q, welche benachbart zu den den Feldern p, auf denen sich ein Loch befindet, sind, ist ein Luftzug spürbar.
			
			\item
			Es gibt ein Feld p, auf dem Gold liegt, ein Feld q, auf dem sich ein Loch befindet und das neben p liegt, sowie ein Feld r, das ebenfalls neben p liegt und auf dem sich der Wumpus befindet. Wumpus und Loch können sich nicht auf dem selben Feld befinden.
			
			\item
			Auf allen Löcher, auf denen sich sowohl der Spieler als auch Wumpus befinden, stirbt der Spieler.
		\end{enumerate}
		
		\item
		\begin{enumerate}
			\item
			\begin{equation*}
				\forall x \exists y \text{ hatLuftZug}(x) \land \text{ hatLoch}(y) \Rightarrow  \text{sindNachbarn}(x, y)
			\end{equation*}
			
			\item
			\begin{equation*}
				\exists x \land \text{ hatLuftZug}(x) \land \text{ hatGestank}(x) \land \text{ hatGold}(x)
			\end{equation*}
			
			\item
			\begin{equation*}
				\exists x, y, z \text{ hatLuftZug}(x) \land y \neq z \land \text{ sindNachbarn}(x, y) \land \text{ sindNachbarn}(x, z) \land \text{hatLoch}(y) \land \text{hatLoch}(z)
			\end{equation*}
		\end{enumerate}
	\end{enumerate}
\end{document}