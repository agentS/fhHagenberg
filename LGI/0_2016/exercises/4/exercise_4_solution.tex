\documentclass[a4paper]{article}
\usepackage{ucs}
\usepackage[utf8x]{inputenc}
\usepackage[T1]{fontenc}
\usepackage{german}
\usepackage{a4,ngerman}
\usepackage[ngerman]{babel}
\usepackage{graphicx}
\usepackage[]{cite}
\usepackage{fancyhdr}
\usepackage{amsmath}
\pagestyle{fancy}
\selectlanguage{german}
\usepackage{array}
\usepackage{mathtools}
\usepackage{amssymb}
\cfoot{\textcopyright Lukas Schörghuber, S1610307103}

\begin{document}
	\begin{enumerate}
		\item
		\begin{enumerate}
			\item
			\begin{center}
				\begin{tabular}{|l|c|c|c|}
					\firsthline
						i & $2^{i}$ & $i \cdot (i + 1)$ & $\frac{2^{i}}{i \cdot (i + 1)}$ \\
					\hline
						3 & 8 & 12 & $\frac{8}{12}$ \\
						4 & 16 & 20 & $\frac{4}{5}$ \\
						5 & 32 & 30 & $\frac{16}{15}$ \\
					\hline
				\end{tabular}
				\newline
				\textbf{Summe = $2.5\overline{3}$}
			\end{center}
			
			\item
			\begin{center}
				\begin{tabular}{|l|l|c|c|}
					\firsthline
						i & j & $k \cdot j$ & $\displaystyle\prod k \cdot j$ \\
					\hline
						2 & 2 & 4 & $4$ \\
						3 & 2 & 6 &  \\
						3 & 3 & 9 & 45 \\
						4 & 2 & 8 &  \\
						4 & 3 & 12 &  \\
						4 & 4 & 16 & 1536 \\
					\hline
				\end{tabular}
				\newline
				\textbf{Summe = $1585$}
			\end{center}
			
			\item
			\begin{center}
				\begin{tabular}{|l|l|c|c|}
					\firsthline
						j & k & k + j & $\displaystyle\sum k + j$ \\
					\hline
						1 & 2 & 3 &  \\
						1 & 3 & 4 & 7 \\
						2 & 3 & 5 &  \\
						2 & 4 & 6 & 11 \\
						3 & 4 & 7 &  \\
						3 & 5 & 8 & 15 \\
						4 & 5 & 9 &  \\
						4 & 6 & 10 & 19 \\
					\hline
				\end{tabular}
				\newline
				\textbf{Produkt} = $7 \cdot 11 \cdot 15 \cdot 19 = 21945$
			\end{center}
		\end{enumerate}
		
		\item
		\begin{enumerate}
			\item
			\begin{enumerate}
				\item
				\begin{center}
					\begin{tabular}{|l|c|}
						\firsthline
							k & $4 \cdot k - 20$ \\
						\hline
							3 & -8 \\
							4 & -4 \\
							5 & 0 \\
							6 & 4 \\
							7 & 8 \\
							8 & 12 \\
							9 & 36 \\
							10 & \textbf{20} \\
						\hline
					\end{tabular}
					\newline
					\textbf{Maximum} = 20
				\end{center}
			
				\item
				\begin{center}
					\begin{tabular}{|l|c|}
						\firsthline
							k & $4 \cdot k - 20$ \\
						\hline
							3 & -8 \\
							4 & -4 \\
							\textbf{5} & 0 \\
							6 & 4 \\
							7 & 8 \\
							8 & 12 \\
							9 & 36 \\
							10 & 20 \\
						\hline
					\end{tabular}
					\newline
					\textbf{Maximum} = 5
				\end{center}
			\end{enumerate}
			\clearpage
			
			\item
			\begin{enumerate}
				\item
				\begin{center}
					\begin{tabular}{|l|c|}
						\firsthline
							k & $(k - 5)^{2}$ \\
						\hline
							2 & 9 \\
							3 & 4 \\
							4 & 1 \\
							5 & \textbf{0} \\
							6 & 1 \\
							7 & 4 \\
						\hline
					\end{tabular}
					\newline
					\textbf{Minimum} = 0
				\end{center}
				
				\item
				\begin{center}
					\begin{tabular}{|l|c|}
						\firsthline
							k & $(k - 5)^{2}$ \\
						\hline
							2 & 9 \\
							\textbf{3} & 4 \\
							4 & 1 \\
							5 & 0 \\
							6 & 1 \\
							7 & 4 \\
						\hline
					\end{tabular}
					\newline
					\textbf{Minimum} = 3
				\end{center}
			\end{enumerate}
		\end{enumerate}
		
		\item
		\begin{enumerate}
			\item
			\begin{equation*}
				A \cup B = \{1, \{1, 2\}, \{2, 3\}, 3, 4\}
			\end{equation*}
			\item
			\begin{equation*}
				A \cup C = \{1, \{1, 2\}, 2, 3, 4, \{3, 4\}\}
			\end{equation*}
			\item
			\begin{equation*}
				B \cup C = \{1, 2, 3, \{2, 3\}, 4, \{3, 4\}\}
			\end{equation*}
			
			\item
			\begin{equation*}
				A \cap B = \{3, 4\}
			\end{equation*}
			\item
			\begin{equation*}
				A \cap C = \{3\}
			\end{equation*}
			\item
			\begin{equation*}
				B \cap C = \{1, 3\}
			\end{equation*}
			
			\item
			\begin{equation*}
				A \text{\textbackslash} B = \{\{1, 2\}\}
			\end{equation*}
			\item
			\begin{equation*}
				A \text{\textbackslash} C = \{\{1, 2\}, 4\}
			\end{equation*}
			
			\item
			\begin{equation*}
				B \text{\textbackslash} C = \{\{2, 3\}, 4\}
			\end{equation*}
			\item
			\begin{equation*}
				B \text{\textbackslash} A = \{1, \{2, 3\}\}
			\end{equation*}
			
			\item
			\begin{equation*}
				C \text{\textbackslash} A = \{1, 2, \{3, 4\}\}
			\end{equation*}
			\item
			\begin{equation*}
				C \text{\textbackslash} B = \{2, \{3, 4\}\}
			\end{equation*}
		\end{enumerate}
		
		\item
		\begin{enumerate}
			\item Falsch. Die Menge mit dem Element 2 ist nicht in A vorhanden.
			\item Falsch. Die Menge mit dem Element 2 ist keine Teilmenge von A, da A auf der obersten Ebene das Element 2 nicht enthält.
			\item Wahr. Das Element 3 ist sowohl in A als auch in C enthalten.
			\item Wahr. Das Element 1 der Menge B ist ein Element des Elementes $\{1, 2\}$ von A.
		\end{enumerate}
		\clearpage
		
		\item
		\begin{enumerate}
			\item $\{2, 3\}$
			\item $\{\{3, 4\}\}$
			\item $\{3\}$
			\item $\{4\}$
		\end{enumerate}
		
		\item
		\begin{enumerate}
			\item $\{p | 20 \leq p \land p \leq 120 \land \text{isPrime}(p)\}$
			\item $\{g | \text{isEven}(g) \land \lnot (\exists x, y \text{ } g = x^{2} + y^{2})\}$
			\item $\{p | \text{isPrime}(p) \land 100 \leq p^{2} \land p^{2} \leq 1000\}$
		\end{enumerate}
		
		\item
		\textbf{zu zeigen}: $\forall M M \neq \{\} \Rightarrow \{\} \subseteq M$
		\newline
		\textbf{zu zeigen}: $\{\} \subseteq M \rightarrow$ Teilmengenaussage: daher in Definition der Teilmenge einsetzen
		\begin{equation*}
			x \subseteq y : \Leftrightarrow \forall z \text{ } z \in x \Rightarrow z \in y
		\end{equation*}
		\textbf{zu zeigen}: $\{\} \subseteq M :\Leftrightarrow \forall z \text{ } z \in \{\} \Rightarrow z \in M$
		\newline
		sei z beliebig, aber fix
		\newline
		\textbf{zu zeigen}: $z \in \{\} \Rightarrow z \in M$
		\newline
		Falls z kein Element der leeren Menge ist, dann ist die Prämisse falsch und da es sich um eine Implikation handelt ist somit die Aussage wahr.
		\newline
		Falls z ein Element der leeren Menge ist, muss die Konklusion $z \in M$ wahr sein.
		\newline
		\textbf{wir wissen}: $z \in M$
		\newline
		\textbf{zu zeigen}: $z \in \{\}$
		\begin{equation*}
			s \in \{y_{0}, y_{1}, ..., y_{n}\} :\Leftrightarrow s = y_{0} \lor s = y_{1} \lor ... \lor s = y_{n}
		\end{equation*}
		\textbf{wir wissen}: $z = M$
		\newline
		\textbf{zu zeigen}: $z \in \{\} \equiv z = \{\}$
	\end{enumerate}
\end{document}
